\documentclass[a4paper,11pt,oneside]{article}
\usepackage[utf8]{inputenc}
%\usepackage{t1enc}
%\usepackage[hungarian]{babel}
\usepackage{url}
\usepackage{sectsty}
\usepackage{enumitem}
\usepackage{array}
\usepackage{longtable}
\usepackage{amsfonts}
\usepackage{pifont}
\usepackage{fontawesome}
\usepackage[usenames,dvipsnames]{color}
\usepackage{hyperref}
\usepackage[normalem]{ulem}

\hypersetup{
    %bookmarks=false,        % show bookmarks bar?
    unicode=true,           % non-Latin characters in Acrobat’s bookmarks
    pdfnewwindow=true,      % links in new window
    colorlinks=true,        % false: boxed links; true: colored links
    linkcolor=MidnightBlue, % color of internal links
    citecolor=green,        % color of links to bibliography
    filecolor=magenta,      % color of file links
    urlcolor=MidnightBlue   % color of external links
}

\newcommand{\cmark}{\ding{51}}%
\newcommand{\xmark}{\ding{55}}%
\allsectionsfont{\sffamily}

%\newcommand{\CC}[1]{\textsf{\small #1}}% to quote from Computing Curricula 2001
\newcommand{\CC}[1]{#1}

\newcommand{\cincl}{{\small\cmark}}
\newcommand{\cdefi}{{\small\cmark\faFileTextO}}
\newcommand{\ccode}{{\small\cmark\faFileText}}
\newcommand{\cnfoc}{{\small\faQuestion}}
\newcommand{\cemay}{{\small\xmark\faQuestionCircle}}
\newcommand{\cexcl}{{\small\xmark}}

\newcommand{\Iincluded}{\item[\hbox to 1.8em{\cincl\hfill}]}
\newcommand{\Idefine}{\item[\hbox to 1.8em{\cdefi\hfill}]}
\newcommand{\Icodeonly}{\item[\hbox to 1.8em{\ccode\hfill}]}
\newcommand{\Inofocus}{\item[\hbox to 1.8em{\cnfoc\hfill}]}
\newcommand{\Iexmaybe}{\item[\hbox to 1.8em{\cemay\hfill}]}
\newcommand{\Iexcluded}{\item[\hbox to 1.8em{\cexcl\hfill}]}

\newcolumntype{C}{ >{\centering\arraybackslash} m{1.2cm} }
\newcommand{\ctable}[1]{
    \begin{center}
        \begin{longtable}{ | C | C | C | m{8cm} | } % TODO: utolsó oszlop ne legyen sorkizárt
        \hline
        \textbf{NT1} & \textbf{NT2} & \textbf{OKTV} & \multicolumn{1}{|c|}{\textbf{Leírás}} \\ \hline
        \endhead
        #1
        \end{longtable}
    \end{center}
}

\newcommand{\DRAFT}{ -- Piszkozat 2020/21-re}
\pagestyle{myheadings}
% TODO: Legyen szélesebb a szöveg
\textwidth 126.34 mm
\textheight 189.55 mm
\parindent 20 pt
\parskip 0 pt
\markboth{\sc Nemes Tihamér Syllabus\DRAFT}{\sc Nemes Tihamér Syllabus\DRAFT}

\newenvironment{myitemize}{\begin{quote}\begin{itemize}\itemsep 0pt}{\end{itemize}\end{quote}}

\newcommand{\remove}[1]{\sout{#1}}
\newcommand{\new}[1]{{\bf \color{red}{#1}}}
%\newcommand{\richard}[1]{}


\begin{document}
\title{\sf Nemes Tihamér Syllabus \\
    \large A Nemes Tihamér NITV Programozás kategória anyagának kivonata}
\author{}
\date{~}
\maketitle

\section{Verzió és státusz}%{{{

Ez egy nem hivatalos javaslat a 2020/2021. tanévi Nemes Tihamér NITV Programozás kategória
második és harmadik fordulójában szereplő feladatok témaköreinek leírására.

\medskip

Jelen verzió még szerkesztés alatt áll, nem teljes, és nem megosztásra szánt.

\medskip

A Nemes Tihamér Syllabus (továbbiakban NT Syllabus) az IOI Syllabus mintájára készül,
és jelenleg még fejlesztés alatt áll. Ha eléri célját, egy hivatalos dokumentum alakul ki belőle,
amelynek aktuális verzióját a versenybizottság hagyja jóvá és teszi közzé minden évben.
Az évek során a dokumentum változhat, fejlődhet.

%}}}

\section{Szerzők és elérhetőségeik}%{{{

Szívesen fogadunk bármilyen visszajelzést és javaslatot a Syllabus-szal kapcsolatban
a jelenlegi szerkesztő e-mail címére küldve (\verb!laszlo.nikhazy@gmail.com!).

Azok számára, akik szeretnének hozzájárulni a dokumentum fejlesztéséhez, vagy hozzászólni,
javaslatokat tenni a tartalmához, az NT Syllabus GitHub repository-jában találhatók információk.
Minden javaslatot, hozzászólást és segítséget szívesen fogadunk.
\url{https://github.com/niklaci/NT-Syllabus}.
%}}}

\section{Bevezetés}\label{sec:intro}%{{{

Az NT Syllabus az IOI Syllabus-hoz képest erősen rövidített, az áttekinthetőség kedvéért.
A konkrét tárgyi ismereteket és módszereket soroljuk fel, míg a készségekről (például hibakeresés)
és eszközök használatáról (például fejlesztőkörnyezet) nem teszünk említést.

\textbf{Ez a dokumentum kifejezetten a 2. és 3. forduló (gépes) feladatairól szól}.
Az első fordulóban szélesebb körből fordulhatnak elő feladatok, mert ott elvárt egy-egy
(esetleg ismeretlen) számítástechnikai témakör alapszintű megértése a feladatleírás alapján.

A Syllabus az alábbi célokat szolgálja.

\begin{itemize}
\item
Meghatározza a versenyre szükséges előzetes tudást.

\item
Segít a versenyzők felkészülésében, a tanároknak a felkészítés tervezésében.

\item
A feladat kitűzőknek támpontot ad, hogy egy-egy feladat melyik kategóriában lehet.

\end{itemize}

A fenti célok elérése érdekében az NT Syllabus-ban megtalálható minden olyan téma
(algoritmusok, adatszerkezetek, módszerek, nyelvi eszközök), amely egyáltalán szóba jöhet a versenyen,
és ezeket kategóriákba soroljuk. A kategóriák tükrözik azt, hogy milyen módon fordulhat elő egy témakör.
Az IOI Syllabus-nak kategóriákat alkalmazunk, az alábbiakban felsoroljuk, majd bővebben magyarázzuk őket.
A verseny három korcsoportjában különböző kategóriába sorolhatók a témakörök.

\begin{myitemize}
\Iincluded Lehet, megkötések nélkül
\Idefine   Lehet, de magyarázandó
\Icodeonly Lehet, de a feladatleírásban nem
\Inofocus  Hatáskörön kívüli
\Iexmaybe  Nem lehet, de nyitott a megvitatásra
\Iexcluded Kizárt
\end{myitemize}


%}}}

\section{Kategóriák}\label{sec:categories}%{{{

This Syllabus classifies a selection of topics into six different ca\-tegories. 
Obviously, such a set of topics can never be exhaustive.
Instead, the list given in this Syllabus should serve as examples that map out the boundary.
Topics not explicitly mentioned in the Syllabus should be classified as follows:
\begin{itemize}
\itemsep -3pt
\item Anything that is a prerequisite of an Included topic is also Included.
\item Anything that is an extension of an Excluded topic or similar to an Excluded topic is also Excluded.
\item Anything else that is not mentioned in the Syllabus is considered Outside of focus.
\end{itemize}


If there is a particular topic for which you are not sure how it should
be classified, we invite you to submit a clarification request to the 
current Syllabus maintainer.

\bigskip

\noindent
A hat lehetséges besorolás:

\begin{description}
\item[\cincl\ Lehet, megkötések nélkül]~\\
    Ebben a kategóriában lévő témakörök előzetes tudásnak tekintendők. A versenyzőktől elvárt az ismeretük.
    A feladatleírásban magyarázat nélkül előfordulhatnak.

    Példa: \emph{Egész típus} ~\S\ref{subsubsec:PF}~PA1

\item[\cdefi\ Lehet, de magyarázandó]~\\
    A versenyzőknek ismerniük kell ezt a témakört, de amikor feladatleírásban szerepel, akkor definiálni kell.
    Általában akkor használjuk ezt a besorolást, ha egy \cincl\  témát többféleképpen is lehet értelmezni.

    Példa: \emph{Feszítőfa} ~\S\ref{subsubsec:DS}~DS5

\item[\ccode\ Lehet, de a feladatleírásban nem]~\\
    Olyan témakörök sorolhatók ide, amelyeknek ismeretére a versenyzőknek a feladatmegoldás során van szüksége.
    Tehát a feladat szövegében nem szerepelhetnek.
    
    Példa: \emph{Aszimptotikus felső becslés a komplexitásra}
    in \S\ref{subsubsec:AL}~AL1

    Ez egy központi téma a verseny kapcsán, hiszen egy helyes program az algoritmus komplexitása alapján kap pontot.
    Mégsem fordulhat elő a feladatleírásban a komplexitás fogalma, a precíz definíció ismerete nem is elvárt.

    It should be noted that this set of topics
    contains a wide range of difficulties, starting from simple concepts and ending 
    with topics that can appear in problems that aim to distinguish among 
    the gold medallists. It is \textbf{not} expected that all contestants
    should know everything listed in this category.

\item[\cnfoc\ Hatáskörön kívül]~\\
    Bármilyen téma, amit nem említ a Syllabus, ebbe a kategóriába esik.
    A versenyzőktől nem elvárt, hogy ismerjék ezeket. A legtöbb versenyfeladat
    nem kapcsolódik ilyen témakörökhöz. De ez nem zárja ki, hogy a versenyen
    legyen olyan feladat, ami ilyen témakörhöz kapcsolódik. A versenybizottság
    kitűzhet ilyen feladatot a verseny színesítése érdekében, de ebben az esetben
    is megoldhatónak kell lennie a fenti kategóriákba eső ismeretekkel.   


\item[\cexcl\ Nem lehet, de nyitott a megvitatásra]~\\
    A Syllabus az évek során változik, fejlődik a versennyel együtt, így lehetőség
    adódik arra, hogy egy (valamely korcsoportban) kizárt téma később bekerüljön
    a verseny anyagába. Általában ezek a témakörök kapcsolódnak megengedett témakörökhöz,
    és nehéz határt szabni. Ebben az esetben a megvitatásra nyitott kategóriába kerülnek
    ezek a témakörök, ezzel is bíztatva a szakmai közösséget a visszajelzésre a kérdést illetően.
    Ha bekerül az idők során egy témakör, akkor a kizárt kategóriából először ebbe kell kerülnie.
    
    Példa: \emph{Szegmensfa\/} ~\S\ref{subsubsec:AL}~Al4

\item[\cemay\ Kizárt]~\\
    Ebbe a kategóriákba főként nehéz algoritmikus módszerek, bonyolult matematikai fogalmak
    tartoznak. Garantált, hogy nem lesz olyan feladat a versenyen, amelynek megoldásához
    elengedhetetlen egy ilyen témakör ismerete. Szintén kívánatos, hogy ezeknek a témáknak
    az ismerete ne nyújtson utat egyszerűbb, vagy több pontot érő megoldáshoz. Ugyanakkor
    előfordulhat, hogy egy feladat témaköre kapcsolódik egy ilyen témához, de ennek ismerete
    nem segít lényegesen a megoldásban.

    Példa: \emph{Heavy-light decomposition} ~\S\ref{subsubsec:AL}~AL4

\end{description}

\bigskip

\noindent
A dokumentum további része a témaköröket és azok korcsoportonkénti besorolását tartalmazza.
%}}}


\section {Számítástechnika, számítástudomány} % {Computing Science}
\label{subsec:computing-science}

\subsection {Progamozási Alapok (PA)} %{Programming Fundamentals (PF)}
\label{subsubsec:PF}

\subsubsection*{PA1. Alapvető programozási eszközök} % {PF1. Fundamental programming constructs}

\ctable {
    \cincl & \cincl & \cincl & Alapvető szintaxisa és szemantikája egy versenyen megengedett programozási nyelvnek
    \\ \hline % Basic syntax and semantics of a higher-level language
    \cincl & \cincl & \cincl & Változók, típusok, műveletek, kifejezések és értékadás
    \\ \hline % Variables, types, expressions, and assignment
    \cincl & \cincl & \cincl & Elágazások és ciklusok
    \\ \hline % Conditional and iterative control structures
    \cincl & \cincl & \cincl & Függvények és paraméterátadás
    \\ \hline % Functions and parameter passing
    \cincl & \cincl & \cincl & Egyszerű beolvasás és kiírás (A standard input/output ismerete kell.)
    \\ \hline % Simple I/O
    \cemay & \cemay & \cemay & Fájlba írás és olvasás
    \\ \hline
}


\subsubsection*{PA2. Alapvető adatszerkezetek} % {PF3. Fundamental data structures}

\ctable {
    \cincl & \cincl & \cincl & Elemi adattípusok (logikai, egész, karakter)
    \\ \hline % Primitive types (boolean, signed/unsigned integer, character)
    \cincl & \cincl & \cincl & Tömbök
    \\ \hline % Arrays (incl. multicolumn dimensional arrays)
    \cincl & \cincl & \cincl & Két vagy több dimenziós tömbök
    \\ \hline
    \cincl & \cincl & \cincl & Karekterláncok (string) és feldolgozásuk
    \\ \hline % Strings and string processing
    \ccode & \ccode & \ccode & Valós számok használata egyszerű és numerikusan stabil számításokra
    \\ \hline % Elementary use of real numbers in numerically stable tasks
    \ccode & \ccode & \ccode & A valós számok lebegőpontos reprezentációja, pontossági hibák léte.
        \footnote{Amikor csak lehetséges, a lebegőpontos számítások teljes elkerülése az előnyösebb megoldás.} % TODO: footnote nincs sehol.
    \\ \hline % \Icodeonly The floating-point representation of real numbers, the existence of precision issues.
        % \footnote{Whenever possible, avoiding floating point computations completely is the preferred solution.}
    \cemay & \ccode & \cincl & Mutatók (pointer) és referenciák
    \\ \hline % Pointers and references
    \cexcl & \cemay & \cincl & Láncolt adatszerkezetek
    \\ \hline % Linked structures
    \cemay & \cemay & \ccode & Törtszámok használata pontos számításokra
    \\ \hline % \Inofocus Using fractions to perform exact calculations.
    \cnfoc & \cnfoc & \cnfoc & Adatok memóriaképe
    \\ \hline % Data representation in memory
    \cnfoc & \cnfoc & \cnfoc & Dinamikus memóriafoglalás
    \\ \hline % {Heap allocation}
    \cexcl & \cexcl & \cexcl & Nemtriviális számítások lebegőpontos számokon, pontossági hibák kiküszöbölése
    \\ \hline % Non-trivial calculations on floating point numbers, manipulating precision errors
}

\subsubsection*{PA3. Rekurzió} %{PF4. Recursion}

\ctable {
    \cincl & \cincl & \cincl & A rekurzió fogalma
    \\ \hline % The concept of recursion
    \ccode & \cincl & \cincl & Rekurzív matematikai függvények
    \\ \hline % Recursive mathematical functions
    \cincl & \cincl & \cincl & Egyszerű rekurzív függvények (több függvény kölcsönös rekurziója is)
    \\ \hline % Simple recursive procedures (incl. mutual recursion)
    \cemay & \ccode & \ccode & A visszalépéses keresés rekurzív változata
    \\ \hline % Recursive backtracking
    \cexcl & \cemay & \cincl & Oszd meg és uralkodj stratégia
    \\ \hline % Divide-and-conquer strategies
}


\subsection {Algoritmusok és komplexitásuk (AL)} % {Algorithms and Complexity (AL)}
\label{subsubsec:AL}

\subsubsection*{AL1. Algoritmusok elemzése alapszinten} % Basic algorithmic analysis

\ctable {
    \cdefi & \cdefi & \cdefi & Algoritmus specifikáció, előfeltétel, utófeltétel, helyesség, invariánsok
    \\ \hline % Algorithm specification, precondition, postcondition, correctness, invariants
    \ccode & \ccode & \ccode & Aszimptotikus felső becslés a komplexitásra (lehetőleg nem formálisan)
    \\ \hline % Asymptotic analysis of upper complexity bounds (informally if possible)
    \cexcl & \ccode & \cdefi & $\mathcal{O}$ (ordó) jelölés a komplexitásra
    \\ \hline % Big O notation
    \ccode & \ccode & \ccode & Szokásos nagyságrendi osztályok: konstans, logaritmikus, lineáris, $\mathcal{O}(n \log n)$, négyzetes, köbös, exponenciális stb.
    \\ \hline % Standard complexity classes: constant, logarithmic, linear, $\mathcal{O}(n \log n)$, quadratic, cubic, exponential, etc.
    \ccode & \ccode & \ccode & Algoritmusok idő- és tárigényének optimalizálása
    \\ \hline % Time and space tradeoffs in algorithms
    \cexcl & \ccode & \ccode & Amortizált komplexitás elemzése
    \\ \hline % AL8. Amortized analysis
    \ccode & \ccode & \ccode & Hatékonyság mérése empirikusan
    \\ \hline % Empirical performance measurements.
    \cnfoc & \cnfoc & \cnfoc & kis ordó, nagy omega és theta jelölések % TODO: görög betűk
    \\ \hline % Little o, Omega, and Theta notation
    \cnfoc & \cnfoc & \cnfoc & Paraméterek hangolása a futási idő vagy memória csökkentése érdekében.
    \\ \hline % Tuning parameters to reduce running time, memory consumption or other measures of performance
    \cexcl & \cexcl & \cexcl & Átlagos komplexitásra aszimptotikus becslések
    \\ \hline % Asymptotic analysis of average complexity bounds
    \cexcl & \cexcl & \cexcl & Rekurziós összefüggések használata komplexitás elemzéskor
    \\ \hline % Using recurrence relations to analyze recursive algorithms
}

\subsubsection*{AL2. Algoritmikus stratégiák} % AL2. Algorithmic strategies

\ctable {
    \ccode & \ccode & \ccode & Egyszerű ciklustervezéses stratégiák
    \\ \hline % Simple loop design strategies
    \ccode & \ccode & \ccode & Kimerítő keresés (brute force)
    \\ \hline % Brute-force algorithms (exhaustive search)
    \ccode & \ccode & \ccode & Mohó algoritmusok
    \\ \hline % Greedy algorithms
    \ccode & \ccode & \ccode & Dinamikus programozás
    \\ \hline % Dynamic programming
    \ccode & \ccode & \ccode & Rekurzív kiszámítás
    \\ \hline
    \ccode & \ccode & \ccode & Két mutató (2 pointers) technika
    \\ \hline
    \cemay & \ccode & \ccode & Bináris keresés szélsőérték meghatározására
    \\ \hline % AL3a. binary search
    \cemay & \ccode & \ccode & Visszalépéses keresés (backtrack, rekurzív és nem rekurzív is)
    \\ \hline % Backtracking (recursive and non-recursive),
    \cexcl & \ccode & \ccode & Elágazás és korlátozás
    \\ \hline % Branch-and-bound
    \cexcl & \ccode & \ccode & Oszd meg és uralkodj elv
    \\ \hline % Divide-and-conquer
    \cnfoc & \cnfoc & \cnfoc & Heurisztikák
    \\ \hline % Heuristics
    \cnfoc & \cnfoc & \cnfoc & Közelítő algoritmusok
    \\ \hline % Discrete approximation algorithms
    \cnfoc & \cnfoc & \cnfoc & Randomizált algoritmusok
    \\ \hline % Randomized algorithms.
    \cexcl & \cexcl & \cexcl & Klaszterező algoritmusok (pl. $k$-means)
    \\ \hline % Clustering algorithms (e.g. $k$-means, $k$-nearest neighbor)
    \cexcl & \cexcl & \cexcl & Többváltozós függvények szélsőérték keresése numerikus módszerekkel
    \\ \hline %Minimizing multi-variate functions using numerical approaches.
}

\subsubsection*{AL3. Algoritmusok} % AL3a. Algorithms

\ctable {
    \ccode & \ccode & \ccode & Egyszerű számelméleti algoritmusok: számrendszer átváltás, Euklideszi algoritmus (LNKO-ra),
    prímteszt $\mathcal{O}(\sqrt{n})$ osztókereséssel, Eratosztenészi szita, prímfelbontés (osztókereséssel vagy szitával)
    \\ \hline % Simple algorithms involving integers: radix conversion, Euclid's algorithm, primality test by $\mathcal{O}(\sqrt{n})$ trial division,
    % Sieve of Eratosthenes, factorization (by trial division or a sieve),
    \cexcl & \cemay & \ccode & Gyors hatványozás (négyzetre emelésekkel)
    \\ \hline % efficient exponentiation
    \cexcl & \cemay & \ccode & Műveletek tetszőlegesen nagy egész számokkal (összeadás, kivonás, szorzás)
    \\ \hline % Simple operations on arbitrary precision integers (addition, subtraction, simple multiplication)
    % \footnote{The necessity to implement these operations should be obvious from the problem statement.}
    \ccode & \ccode & \ccode & Egyszerű programozási tételek tömbökön: összegzés, megszámolás, keresés, minimum/maximum, kiválogatás
    \\ \hline % Simple array manipulation (filling, shifting, rotating, reversal, resizing, minimum/maximum, prefix sums, histogram, bucket sort)
    % {sequential} processing/search
    \ccode & \ccode & \ccode & Programozási tételek összeépítése, pl. feltételes maximum
    \\ \hline % Simple array manipulation (filling, shifting, rotating, reversal, resizing, minimum/maximum, prefix sums, histogram, bucket sort)
    \ccode & \ccode & \ccode & Rendezett sorozatok összefésülése, metszet, unió
    \\ \hline
    \ccode & \ccode & \ccode & Egyszerű string algoritmusok (pl. minta keresése naiv módszerrel)
    \\ \hline % Simple string algorithms (e.g., naive substring search)
    \ccode & \ccode & \ccode & {$\mathcal{O}(n^2)$} rendezések (buborék, leszámláló, minimum-kiválasztásos)
    \\ \hline
    \cemay & \ccode & \ccode & Gyorsrendezés (quicksort), \textit{NT1-ben a sort() függvény használata elvárt, de az algoritmus ismerete nem}
    \\ \hline % {Quicksort} and Quickselect to find the $k$-th smallest element.
    \cexcl & \ccode & \ccode & $\mathcal{O}(n \log n)$ rendezések (kupac, összefésüléses)
    \\ \hline % {$\mathcal{O}(n \log n)$} worst-case \CC{sorting algorithms (heap sort, merge sort)}
    \cexcl & \cemay & \ccode & Lineáris idejű rendezések (láda/vödör rendezés, radix rendezés)
        \footnote{A gyakorlatban a futási időt lehetetlen megkülönböztetni a beépített rendezéstől, így csak kifejezetten erről szóló feladatnál kell.}
    \\ \hline % bucket sort
    \cemay & \ccode & \ccode & Rekurzív fabejárás
    \\ \hline
    \cexcl & \cdefi & \cdefi & Rendezett fák bejárásai (pre, in- és post-order)
    \\ \hline % Traversals of ordered trees (pre-, in-, and post-order)
    \cexcl & \ccode & \ccode & Szélességi és mélységi gráfbejárás
    \\ \hline % Depth- and breadth-first traversals
    \cexcl & \ccode & \ccode & Összefüggő komponensek meghatározása
    \\ \hline % Finding connected components and transitive closures.
    \cexcl & \ccode & \ccode & A mélységi feszítőfa alkalmazásai, például topologikus sorrend
    \\ \hline % Applications of the depth-first traversal tree, such as topological ordering
    \cexcl & \cdefi & \cdefi & Irányított körmentes gráf emeletekre bontása
    \\ \hline 
    \cexcl & \cdefi & \cdefi & Euler-séta/körséta keresése
    \\ \hline % and Euler paths/cycles
    \cexcl & \cdefi & \cdefi & Legrövidebb utak súlyozott gráfokban (Dijkstra, Bellman-Ford, Floyd-Warshall)
    \\ \hline % Shortest-path algorithms (Dijkstra, Bellman-Ford, Floyd-Warshall)
    \cexcl & \cdefi & \cdefi & Minimális feszítőfa keresése (Prim és Kruskal algoritmus)
    \\ \hline % Minimum spanning tree (Jarn\'\i k-Prim and Kruskal algorithms)
    \cexcl & \cemay & \cdefi & Elvágó pontok, hídélek keresése, kétszeres (pont- ill. él-) összefüggőség
    \\ \hline % Biconnectivity in undirected graphs (bridges, articulation points).
    \cexcl & \cemay & \cdefi & Irányított gráfok összefüggősége, erősen összefüggő komponensek
    \\ \hline % Connectivity in directed graphs (strongly connected components).
    \cexcl & \cexcl & \cdefi & Páros gráfban maximális párosítás keresése magyar módszerrel ($O(VE)$ időben)
    \\ \hline % $O(VE)$ time algorithm for computing maximum bipartite matching.
    \ccode & \cdefi & \cdefi & Kombinatorikus játékok alapjai, nyerő és vesztő pozíciók
    \\ \hline % Basics of combinatorial game theory, winning and losing positions,
    \cexcl & \ccode & \ccode & Minimax algoritmus kétszemélyes játékok optimális stratégiájára
    \\ \hline % minimax algorithm for optimal game playing
    \cexcl & \cexcl & \cexcl & Alfa-béta vágás
    \\ \hline % Alpha-beta pruning
    \cexcl & \cexcl & \cexcl & Hálózati folyamok, maximális folyam és minimális vágás keresése
    \\ \hline % Maximum flow. Flow/cut duality theorem.
    \cexcl & \cexcl & \cexcl & Matroidok és hozzájuk kapcsolódó optimalizálási problémák
    \\ \hline % optimization problems that are easiest to analyze using matroid theory. Problems based on matroid intersecions (except for bipartite matching).
    \cexcl & \cexcl & \cexcl & Lexikografikus szélességi bejárás, maximum adjacency search % TODO: mi ez magyarul, egyáltalán kell-e ?
    \\ \hline % Lexicographical BFS, maximum adjacency search and their properties
    \cexcl & \cexcl & \cemay & Haladó string algoritmusok: KMP, Z-algoritmus
    \\ \hline % AL3b. String algorithms and data structures: there is evidence that the inter-reducibility between KMP, Rabin-Karp hashing, suffix arrays/tree, suffix automaton, and Aho-Corasick makes them difficult to separate. 
}

\subsubsection*{AL4. Adatszerkezetek (összetett adatszerkezetek)} % AL3b. Data structures

\ctable {
    \ccode & \cdefi & \cincl & Verem és sor
    \\ \hline % Stacks and queues
    \ccode & \cdefi & \cincl & Prefix összeg (kumulatív összeg)
    \\ \hline % Al3a. prefix sums
    \ccode & \ccode & \ccode & Számláló tömb, hisztogram
    \\ \hline % Al3a. histogram
    \cemay & \cdefi & \cincl & Gráfreprezentációk: mátrix, szomszédsági lista, éllista
    \\ \hline % Representations of graphs (adjacency lists, adjacency matrix
    \cexcl & \ccode & \cincl & Unió-holvan (DSU)
    \\ \hline % Representation of disjoint sets: the Union-Find data structure.
    \cexcl & \cdefi & \cdefi & Kupac és hasonló bináris fa reprezentációk
    \\ \hline % Binary heap data structures
    \cexcl & \cdefi & \cincl & Prioritási sor használata (kupac ismerete nélkül)
    \\ \hline 
    \cexcl & \ccode & \ccode & Beépített set és map használata
    \\ \hline 
    \cexcl & \cexcl & \cemay & Szegmensfa, Fenwick-fa, és hasonló statikusan kiegyensúlyozott bináris keresőfák
    \\ \hline % Statically balanced binary search trees. Instances of this include binary index trees (also known as Fenwick trees)
        % and segment trees (also known as interval trees and tournament trees).
        % \footnote{Not to be confused with similarly-named data structures used in computational geometry.}
    \cexcl & \cexcl & \cexcl & Kiegyensúlyozott bináris keresőfák (pl. AVL-fa, piros-fekete fa stb.)
        \footnote{Ugyan nem kell ismerni ezeket az adatszerkezeteket, de az erre épülő sztenderd tárolók (set, map) használata szükséges lehet.}
    \\ \hline % {Balanced binary search trees} \footnote{Problems will not be designed to distinguish between
        % the implementation of BBSTs, such as treaps, splay trees, AVL trees, or scapegoat trees}
    % TODO: Augmented binary search trees - ez mi???
    \cexcl & \cexcl & \cemay & LCA (legalacsonyabb közös ős fában) meghatározása $O(\log n)$ időben.
    \\ \hline % $O(\log n)$ time algorithms for answering lowest common ancestor queries in a static rooted tree.
        % \footnote{Once again, different implementations meeting this requirement will not be distinguished.}
    \cexcl & \cemay & \ccode & Adatszerkezetek egymásba ágyazása (például halmazok halmaza).
    \\ \hline % Nesting of data structures, such as having a sequence of sets.
    \cexcl & \cexcl & \cdefi & Szófa (trie)
    \\ \hline % Tries
    \cexcl & \cexcl & \cexcl & Összetett string adatszerkezetek (suffix arrays/tree, suffix automaton, Rabin-Karp hashing)
    \\ \hline % String algorithms and data structures: there is evidence that the inter-reducibility between KMP, Rabin-Karp hashing, suffix arrays/tree, suffix automaton, and Aho-Corasick makes them difficult to separate.
    \cexcl & \cexcl & \cexcl & Perzisztens adatszerkezetek (például perzisztens szegmensfa) % TODO: tényleg ezt jelenti az alábbi ??
    \\ \hline % Creating persistent data structures by path copying.
    \cexcl & \cexcl & \cexcl & Heavy-light decomposition
    \\ \hline % Heavy-light decomposition and separator structures for static trees.
    \cexcl & \cexcl & \cexcl & Dinamikusan változó fákhoz fejlett adatszerkezetek
    \\ \hline % Data structures for dynamically changing trees and their use in graph algorithms.
    \cexcl & \cexcl & \cexcl & Bonyolult kupac-variációk (Fibonacci, binomiális)
    \\ \hline % Complex heap variants such as binomial and Fibonacci heaps,
    \cexcl & \cexcl & \cexcl & Hash-táblák implementálása, illetve használata
        \footnote{Természetesen lehet használni a beépített implementációkat, de nem lehet olyan feladat, ami enélkül nem oldható meg.}
    \\ \hline % Using and implementing \CC{hash tables} (incl. strategies to resolve collisions)
    \cexcl & \cexcl & \cexcl & Két dimenziós fa alapú adatszerkezetek (pl. 2D Fenwick-fa)
    \\ \hline % Two-dimensional tree-like data structures (such as a 2D statically balanced binary tree or a treap of treaps) used for 2D queries.
}


\subsection*{AL5. Geometriai algoritmusok } % AL10. Geometric algorithms

\begin{quote}
    A versenybizottság álláspontja, hogy olyan problémák szerepeljenek a versenyen, amelyek
    egész számokkal való számításokkal megoldhatók. Lebegőpontos számításokat igénylő feladatok
    bevonása a versenyre meggondolandó a jövőben, ha szakmailag indokolt.
    % In general, the ISC has a strong preference towards problems that can be solved using integer
    % arithmetics to avoid precision issues. This may include representing some computed values as 
    % exact fractions, but extensive use of such fractions in calculations is discouraged.
    % Additionally, if a problem uses two-dimensional objects, the ISC prefers problems in which such objects are rectilinear.
\end{quote}

\ctable {
    \cexcl & \cemay & \cincl & Pontok, szakaszok, egyenesek és vektorok reprezentálása
    \\ \hline % Representing points, vectors, lines, line segments. 
    \cexcl & \cemay & \cincl & Forgásirány meghatározása, párhuzamos és merőleges vektorok, egy egyenesen lévő pontok
    \\ \hline % Checking for collinear points, parallel/orthogonal vectors and clockwise turns (for example,
    % by using dot products and cross products).
    \cexcl & \cemay & \cincl & Pontok távolságának, vektorok hosszának összehasonlítása (távolságnégyzet, gyökvonás nélkül)
    \\ \hline % 
    \cexcl & \cemay & \cincl & Szakaszok metszéspontja
    \\ \hline % Intersection of two lines.
    \cexcl & \cemay & \cincl & Egy konvex/konkáv sokszögön belül vagy kívül van-e egy pont
    \\ \hline % Checking whether a (general/convex) polygon contains a point. 
    \cexcl & \cexcl & \cemay & Sokszög területe a csúcsok koordinátái alapján
    \\ \hline % Computing the area of a polygon from the coordinates of its vertices.
    % \footnote{The recommended way of doing so is to use cross products or an equivalent formula. TODO url} 
    \cexcl & \cexcl & \ccode & Koordináta tömörítés 
    \\ \hline % Coordinate compression.
    \cexcl & \cexcl & \ccode & Sarokponthoz képesti polárszög szerinti rendezés
    \\ \hline % 
    \cexcl & \cexcl & \ccode & Söprés, söprő egyenes módszer
    \\ \hline % Sweeping line method
    \cexcl & \cexcl & \cdefi & Konvex burok meghatározása $\mathcal{O}(n\log{n})$ algoritmussal
    \\ \hline % $\mathcal{O}(n\log{n})$ time algorithms for convex hull
    \cexcl & \cexcl & \cexcl & Kör metszéspontja egyenessel illetve körrel
    \\ \hline % Computing coordinates of circle intersections against lines and circles.
    \cexcl & \cexcl & \cexcl & Síkbeli alakzat súlypontjának kiszámítása
    \\ \hline % Center of mass of a 2D object.
    \cexcl & \cexcl & \cexcl & Geometriai transzformációk lineáris algebrai reprezentációja
    \\ \hline % Computing and representing the composition of geometric transformations if the knowledge of linear algebra gives an advantage. 
    \cexcl & \cexcl & \cexcl & Lineáris programozás és geometriai interpretációja
    \\ \hline % Linear programming in 3 or more dimensions and its geometric interpretations. 
}



\subsubsection*{AL6. Egyéb algoritmuselméleti területek}%{{{

\ctable {
    \cexcl & \cexcl & \cemay & Online algoritmusok
    \\ \hline % AL8. Advanced algorithmic analysis -> Online algorithms
    \cnfoc & \cnfoc & \cnfoc & Randomizált algoritmusok
    \\ \hline % AL7. AL8. Advanced algorithmic analysis -> Randomized algorithms
    \cexcl & \cexcl & \cemay & Automaták és nyelvtanok, reguláris kifejezések
    \\ \hline % AL7. Automata and grammars
    \cnfoc & \cnfoc & \cnfoc & Megállási probléma, nem kiszámítható függvények
    \\ \hline %AL5. Basic computability {Uncomputable functions}; {The halting problem};
    \cnfoc & \cnfoc & \cnfoc & P és NP komplexitási osztályok
    \\ \hline % AL6. The complexity classes P and NP
    \cnfoc & \cnfoc & \cnfoc & Elosztott algoritmusok
    \\ \hline % AL4. Distributed algorithms
    \cnfoc & \cnfoc & \cnfoc & Kriptográfiai algoritmusok, titkosítás
    \\ \hline % AL9. Cryptographic algorithms
}

\section {Matematika} % Mathematics
\label{subsec:mathematics}

\subsection{Aritmetika és geometria} % {Arithmetics and Geometry}
\label{subsubsec:NG}

\ctable {
    \cincl & \cincl & \cincl & Egész számok, alapműveletek (hatványozás is), összehasonlítás
    \\ \hline %  Integers, operations (incl.\ exponentiation), comparison
    \cincl & \cincl & \cincl & Egész számok alapvető tulajdonságai: előjel, paritás, oszthatóság
    \\ \hline %  Basic properties of integers (sign, parity, divisibility)
    \cincl & \cincl & \cincl & Törtek, százalékok
    \\ \hline %  Fractions, percentages
    \cincl & \cincl & \cincl & Számrendszerek, mértékegység átváltások (naptár ismerete)
    \\ \hline %
    \cdefi & \cdefi & \cdefi & Számtani és mértani sorozatok 
    \\ \hline %
    \cincl & \cincl & \cincl & Alapvető geometriai fogalmak: egyenes, szakasz, szög, sokszög,
    háromszög, téglalap, négyzet, kör
    \\ \hline %  Line, line segment, angle, triangle, rectangle, square, circle
    \cexcl & \cemay & \cincl & Pont, vektor, síkbeli koordináták
    \\ \hline %  Point, vector, coordinates in the plane
    \cexcl & \cemay & \cincl & Sokszögekkel kapcsolatos fogalmak: csúcs, oldal, konvex/konkáv, kerület, terület, tartalmazás
    \\ \hline %  Polygon (vertex, side/edge, simple, convex, inside, area)
    \cexcl & \cdefi & \cdefi & Euklideszi távolság 
    \\ \hline % Prime numbers
    \cexcl & \ccode & \ccode & Pitagorasz-tétel 
    \\ \hline % Pythagorean theorem
    \cexcl & \cexcl & \cexcl & Geometria 3D (vagy több dimenziós) térben 
    \\ \hline % geometry in 3D or higher dimensional spaces
    \cexcl & \cexcl & \cexcl & Lebegőpontos számítások pontosságának elemzése és javítása 
    \\ \hline % analyzing and increasing precision of floating-point \\ computations
    \cexcl & \cexcl & \cexcl & Komplex számok
    \\ \hline % complex numbers,
    \cexcl & \cexcl & \cexcl & Kúpszeletek (parabola, hiperbola, ellipszis) 
    \\ \hline % general conics (parabolas, hyperbolas, ellipses)
    \cexcl & \cexcl & \cemay & Trigonometriai függvények 
    \\ \hline % trigonometric functions
}

\subsection{Számelmélet} % {Arithmetics and Geometry részen belül van}
\label{subsubsec:NT}

\ctable {
    \cincl & \cincl & \cincl & Osztó-többszörös viszony
    \\ \hline %
    \cincl & \cincl & \cincl & Műveletek maradékokkal 
    \\ \hline %  Basic modular arithmetic: addition, subtraction, \\ multiplication
    \cdefi & \cincl & \cincl & Prímszámok 
    \\ \hline % Prime numbers
    \cemay & \cdefi & \cdefi & Prímtényezős felbontás 
    \\ \hline %
    \cincl & \cincl & \cincl & Legnagyobb közös osztó, legkisebb közös többszörös 
    \\ \hline %
    \cexcl & \cexcl & \cemay & Modulo inverz 
    \\ \hline % \Iexcluded modular division and inverse elements
    \cexcl & \cemay & \cemay & További számelméleti témák 
    \\ \hline % Additional topics from number theory.
    
}


\subsection {Véges matematika} % {Discrete Structures (DS)}
\label{subsubsec:DS}

\subsubsection*{VM1. Halmazok, relációk és függvények} % DS1. Functions, relations, and sets

\ctable{
    \cexcl & \ccode & \cdefi & Függvények tulajdonságai (kompozíció, kölcsönös egyértelműség, szürjekció, injekció, inverz) 
    \\ \hline % \Idefine\CC{Functions (surjections, injections, inverses, composition)}
    \cexcl & \cdefi & \cdefi & Relációk tulajdonságai (szimmetrikus, tranzitív, reflexív, ekvivalenciareláció)
    \\ \hline % \Idefine\CC{Relations (reflexivity, symmetry, transitivity, equivalence relations)}
    \cdefi & \cincl & \cincl & Rendezési reláció, lexikografikus sorrend
    \\ \hline % total/linear order relations, lexicographic order
    \cincl & \cincl & \cincl & Halmazok: eleme, részhalmaz, unió, metszet
    \\ \hline % Sets (inclusion/exclusion,
    \cexcl & \cdefi & \cdefi & Halmazok: komplementer, Descartes-szorzat, hatványhalmaz
    \\ \hline % complements, Cartesian products, power sets)}
    \cexcl & \cexcl & \cexcl & Végtelen halmazok számossága
    \\ \hline % {Cardinality and countability} (of infinite sets)

}

\subsubsection*{VM2. Matematikai logika} % {DS2. Basic logic}

\ctable {
    \cincl & \cincl & \cincl & Logikai műveletek és tulajdonságaik (nem, és, vagy, ha-akkor)
    \\ \hline  % {Logical connectives} (incl.\ their basic properties)
    \cincl & \cincl & \cincl & Igazságtáblák
    \\ \hline  % {Truth tables}
    \ccode & \ccode & \cdefi & Minden és létezik (a kvantorral való jelölés elkerülendő)
    \\ \hline  % {Universal and existential quantification}
    \cexcl & \ccode & \ccode & Elsőrendű logika, modus ponens
    \\ \hline  % First-order logic
    \cexcl & \cexcl & \cemay & Normál formák
    \\ \hline  % \Inofocus \CC{Normal forms}
    \cexcl & \cexcl & \cexcl & A predikátumlogika limitációi
    \\ \hline % Limitations of predicate logic
}

\subsubsection*{VM3. Bizonyítási módszerek} % DS3. Proof techniques

\ctable{
    \cincl & \cincl & \cincl & Indoklás ellenpéldával
    \\ \hline  % {Direct proofs, proofs by: counterexample}
    \ccode & \ccode & \ccode & Indirekt bizonyítás
    \\ \hline  %
    \cexcl & \cemay & \cdefi & Implikáció fogalma, ekvivalens formája, tagadása
    \\ \hline  % {Notions of implication, converse, inverse, contrapositive, negation, and contradiction}
    \cexcl & \ccode & \ccode & Matematikai indukció, teljes indukció
    \\ \hline  %
    \ccode & \ccode & \cdefi & Rekurzív összefüggések
    \\ \hline  % {Recursive mathematical definitions} (incl.\ mutually recursive definitions)    
}

\subsubsection*{VM4. Kombinatorika: leszámlálások} % DS4. Basics of counting

\ctable{
    \ccode & \ccode & \ccode & Szorzási szabály (független választások)
    és összeadási szabály (egymást kizáró választások) leszámlálásokban
    \\ \hline  % {Counting arguments (sum and product rule,
    \ccode & \ccode & \ccode & Számtani és mértani sorozatos növekedés
    \\ \hline  % arithmetic and geometric progressions,
    \ccode & \cdefi & \cdefi & Fibonacci-típusú és egyéb rekurzív sorozatok
    \\ \hline  % Fibonacci numbers
    \cdefi & \cincl & \cincl & Permutációk, kombinációk és variációk
    \\ \hline  % {Permutations and combinations (basic definitions)}
    \ccode & \ccode & \ccode & Logikai szita (sok halmazra csak OKTV-ben)
    \\ \hline  % {Inclusion-exclusion principle} 
    \ccode & \ccode & \ccode & Skatulyaelv
    \\ \hline  % {Pigeonhole principle}
    \cemay & \ccode & \ccode & Pascal-háromszög, binomiális tétel
    \\ \hline  % {Pascal's identity}, \CC{Binomial theorem}
    \ccode & \ccode & \ccode & Módszerek lexikografikusan következő/előző
    és N-edik elem előállítására
    \\ \hline  %
    \cexcl & \cexcl & \cexcl & Rekurzív formulák explicit alakra hozása
    \\ \hline  % \Iexcluded  Solving of recurrence relations
    \cexcl & \cexcl & \cexcl & Burnside-lemma
    \\ \hline  % \Iexcluded Burnside lemma
}



\subsubsection*{VM5. Gráfelmélet} % {DS5. Graphs and trees}

    \begin{quote}
        Az NT1 korcsoportban implicit szerepelhetnek a gráfok egy-egy feladatban, de nem
        szükséges hozzá a fogalmak ismerete, illetve alapelv, hogy ez ne okozzon lényeges előnyt.
    \end{quote}

\ctable{
    \ccode & \cdefi & \cincl & Fa gráfok és tulajdonságaik, gyökeres fa
    \\ \hline  % {Trees} and their basic properties, rooted trees
    \ccode & \ccode & \cdefi & Irányítatlan gráf és hozzá kapcsolódó fogalmak:
    út, kör, fokszám, összefüggőség, komponens
    \\ \hline  % {Undirected graphs} (degree, path, cycle, connectedness)
    \cexcl & \ccode & \ccode & Euler- és Hamilton-út/kör irányítatlan és irányított gráfban is
    \\ \hline  % Euler/Hamil\-ton path/cycle, handshaking lemma -> TODO: ez mi??
    \ccode & \ccode & \ccode & Irányított gráf és hozzá kapcsolódó fogalmak:
    irányított út és kör, be- és kifok
    \\ \hline  % {Directed graphs} (in-degree, out-degree, directed path/cycle, Euler/Hamilton path/cycle)
    \cexcl & \cdefi & \cdefi & Súlyozott, illetve színezett gráfok
    \\ \hline  % `Decorated' graphs with edge/node labels, weights, colors
    \ccode & \cdefi & \cdefi & Párhuzamos- és hurokélek
    \\ \hline  % 
    \cexcl & \cdefi & \cdefi & Feszítőfa, minimális feszítőfa 
    \\ \hline  %  {Spanning trees}
    \cexcl & \cdefi & \cdefi & Gráfbejárási stratégiák
    \\ \hline  % {Traversal strategies}
    \cexcl & \cemay & \ccode & Többszörös összefüggőség, elvágó pontok, hídélek
    \\ \hline  % {Traversal strategies}
    \cexcl & \cdefi & \cdefi & Páros gráf
    \\ \hline  % {Bipartite graphs}
    \cexcl & \cemay & \ccode & Erősen összefüggőség (irányított gráf)
    \\ \hline  % {Bipartite graphs}
    \cexcl & \ccode & \ccode & Síkbarajzolható gráf (lehet olyan feladat, amiben ez a feltétel teljesül a gráfra)
    \\ \hline  % Planar graphs
    \cexcl & \cexcl & \cexcl & Síkbarajzolhatóság eldöntése, Kuratowski-tétel
    \\ \hline  % Planarity testing
    \cexcl & \cexcl & \cexcl & Hipergráfok
    \\ \hline  % Hypergraphs
    \cexcl & \cexcl & \cexcl & Speciális gráf osztályok, mint például perfekt gráfok
    \\ \hline  % Specific graph classes such as perfect graphs
    \cexcl & \cexcl & \cemay & Hálózati folyamok
    \\ \hline  % 
}

\subsection {A matematika egyéb területei} % {Other Areas in Mathematics}
\label{subsubsec:other-mathematics}

\ctable {
    \cnfoc & \cnfoc & \cnfoc & Valószínűségszámítás (a diszkrét valószínűség könnyen
    átalakítható kombinatorika feladatokra)
    \\ \hline  % DS6. Discrete probability
    \cexcl & \cexcl & \cexcl & Térgeometria, és magasabb dimenziós geometria
    \\ \hline  % Geometry in three or more dimensions.
    \cexcl & \cexcl & \cexcl & Lineáris algebra, például vektorterek, mátrix műveletek,
    Gauss-elimináció, FFT
    \\ \hline  % Linear algebra, including (but not limited to): Matrix multiplication, exponentiation, 
    % inversion, and Gaussian elimination, Fast Fourier transform
    \cexcl & \cexcl & \cexcl & Analízis
    \\ \hline  % Calculus
    \cexcl & \cexcl & \cexcl & Haladó kombinatorikus játékelmélet, például NIM és Sprague-Grundy tétel
    \\ \hline  % Theory of combinatorial games, e.g., NIM game, Sprague-Grundy theory
    \cexcl & \cexcl & \cexcl & Statisztika
    \\ \hline  % Statistics

}

\section {Szoftvefejlesztés} % {Software Engineering (SE)}
\label{subsec:software-engineering}

\begin{quote}
    Az NT Syllabus-ban kisebb súllyal jelenik meg ez a téma, mint az IOI Syllabus-ban, mivel
    elsősorban algoritmizálásról és ehhez kapcsolódó fogalmakról szól ez a dokumentum. A versenyen
    sokkal kevésbé fontos a szoftvertervezés, mint egy összetett alkalmazás fejlesztésében.
\end{quote}

\ctable{
    \ccode & \ccode & \ccode & Fekete-doboz tesztelés, be- és kimenet alapján
    \\ \hline  % {Black-box and white-box testing techniques}
    \ccode & \ccode & \ccode & Függvénykönyvtárak használata (például C++ STL)
    \\ \hline  % 
    \cexcl & \cexcl & \cemay & Egyszerű API megvalósítása (például interaktív feladatok)
    \\ \hline  % {API (Application Programming Interface) programming}
    \ccode & \ccode & \ccode & Valamilyen programozási környezet (IDE), fordítóprogam (compiler) ismerete
    \\ \hline  % {Programming environments}, incl.\ IDE (Integrated Development Environment)
    \cexcl & \cexcl & \cexcl & Objektum-orientált programozás és tervezési minták
    \\ \hline  % {Object-Oriented analysis and design}, {Design patterns}
    \cexcl & \cexcl & \cexcl & Szoftver-architektúrák
    \\ \hline  % {Software architecture}
    
}

\section {Digitális írástudás} % {Computer Literacy}
\label{subsec:computer-literacy}

\begin{quote}
    The text of this section is \ccode .
    
    \medskip
    
    Contestants should know and understand the basic structure and operation of a computer
    (CPU, memory, I/O).
    They are expected to be able to use a standard computer with graphical user interface,
    its operating system with supporting applications, and the provided program development tools
    for the purpose of solving the competition tasks. In particular,some skill in file management
    is helpful (creating folders, copying and moving files).
    Typically, some services are available through a standard web browser.    
    It is often the case that a number of equivalent tools are made available.
    The contestants are not expected to know all the features of all these tools.
    They can make their own choice based on what they find most appropriate.
\end{quote}    

\end{document}
