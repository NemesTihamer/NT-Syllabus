\documentclass[a4paper,11pt,oneside]{article}
\usepackage[utf8]{inputenc}
%\usepackage{t1enc}
%\usepackage[hungarian]{babel}
\usepackage{url}
\usepackage{sectsty}
\usepackage{enumitem}
\usepackage{array}
\usepackage{longtable}
\usepackage{amsfonts}
\usepackage{pifont}
\usepackage{fontawesome}
\usepackage[usenames,dvipsnames]{color}
\usepackage{hyperref}
\usepackage[normalem]{ulem}

\hypersetup{
    %bookmarks=false,        % show bookmarks bar?
    unicode=true,           % non-Latin characters in Acrobat’s bookmarks
    pdfnewwindow=true,      % links in new window
    colorlinks=true,        % false: boxed links; true: colored links
    linkcolor=MidnightBlue, % color of internal links
    citecolor=green,        % color of links to bibliography
    filecolor=magenta,      % color of file links
    urlcolor=MidnightBlue   % color of external links
}

\newcommand{\cmark}{\ding{51}}%
\newcommand{\xmark}{\ding{55}}%
\allsectionsfont{\sffamily}

%\newcommand{\CC}[1]{\textsf{\small #1}}% to quote from Computing Curricula 2001
\newcommand{\CC}[1]{#1}

\newcommand{\cincl}{{\small\cmark}}
\newcommand{\cdefi}{{\small\cmark\faFileTextO}}
\newcommand{\ccode}{{\small\cmark\faFileText}}
\newcommand{\cnfoc}{{\small\faQuestion}}
\newcommand{\cemay}{{\small\xmark\faQuestionCircle}}
\newcommand{\cexcl}{{\small\xmark}}

\newcommand{\Iincluded}{\item[\hbox to 1.8em{\cincl\hfill}]}
\newcommand{\Idefine}{\item[\hbox to 1.8em{\cdefi\hfill}]}
\newcommand{\Icodeonly}{\item[\hbox to 1.8em{\ccode\hfill}]}
\newcommand{\Inofocus}{\item[\hbox to 1.8em{\cnfoc\hfill}]}
\newcommand{\Iexmaybe}{\item[\hbox to 1.8em{\cemay\hfill}]}
\newcommand{\Iexcluded}{\item[\hbox to 1.8em{\cexcl\hfill}]}

\newcolumntype{C}{ >{\centering\arraybackslash} m{1.2cm} }
\newcommand{\ctable}[1]{
    \begin{center}
        \begin{longtable}{ | C | C | C | m{8cm} | }
        \hline
        \textbf{NT1} & \textbf{NT2} & \textbf{OKTV} & \multicolumn{1}{|c|}{\textbf{Leírás}} \\ \hline
        \endhead
        #1
        \end{longtable}
    \end{center}
}

\newcommand{\DRAFT}{ -- Piszkozat 2020/21-re}
\pagestyle{myheadings}
% TODO: Legyen szélesebb a szöveg
\textwidth 126.34 mm
\textheight 189.55 mm
\parindent 20 pt
\parskip 0 pt
\markboth{\sc Nemes Tihamér Sillabusz\DRAFT}{\sc Nemes Tihamér Sillabusz\DRAFT}

\newenvironment{myitemize}{\begin{quote}\begin{itemize}\itemsep 0pt}{\end{itemize}\end{quote}}

\newcommand{\remove}[1]{\sout{#1}}
\newcommand{\new}[1]{{\bf \color{red}{#1}}}
%\newcommand{\richard}[1]{}


\begin{document}
\title{\sf Nemes Tihamér Sillabusz \\
    \large A Nemes Tihamér NITV Programozás kategória anyagának kivonata}
\author{}
\date{~}
\maketitle

\section{Verzió és státusz}%{{{

Ez egy nem hivatalos javaslat a 2020/2021. tanévi Nemes Tihamér NITV Programozás kategória
második és harmadik fordulójában szereplő feladatok témaköreinek leírására.

\medskip

Jelen verzió még szerkesztés alatt áll, nem teljes, és nem megosztásra szánt.

\medskip

A Nemes Tihamér Sillabusz (továbbiakban NT Sillabusz) az IOI Syllabus mintájára készül,
és jelenleg még fejlesztés alatt áll. Ha eléri célját, egy hivatalos dokumentum alakul ki belőle,
amelynek aktuális verzióját a versenybizottság hagyja jóvá és teszi közzé minden évben.
Az évek során a dokumentum változhat, fejlődhet.

%}}}

\section{Szerzők és elérhetőségeik}%{{{

Szívesen fogadunk bármilyen visszajelzést és javaslatot a sillabusszal kapcsolatban
a jelenlegi szerkesztő e-mail címére küldve (\verb!laszlo.nikhazy@gmail.com!).

Azok számára, akik szeretnének hozzájárulni a dokumentum fejlesztéséhez, vagy hozzászólni,
javaslatokat tenni a tartalmához, az NT Sillabusz GitHub repository-jában találhatók információk.
Minden javaslatot, hozzászólást és segítséget szívesen fogadunk.
\url{https://github.com/niklaci/NT-Syllabus}.
%}}}

\section{Bevezetés}\label{sec:intro}%{{{

Az NT Sillabusz az IOI Syllabus-hoz képest erősen rövidített, az áttekinthetőség kedvéért.
A konkrét tárgyi ismereteket és módszereket soroljuk fel, míg a készségekről (például hibakeresés)
és eszközök használatáról (például fejlesztőkörnyezet) nem teszünk említést.

\textbf{Ez a dokumentum kifejezetten a 2. és 3. forduló (gépes) feladatairól szól}.
Az első fordulóban szélesebb körből fordulhatnak elő feladatok, mert ott elvárt egy-egy
(esetleg ismeretlen) számítástechnikai témakör alapszintű megértése a feladatleírás alapján.

A Sillabusz az alábbi célokat szolgálja.

\begin{itemize}
\item
It specifies a small set of required prerequisite knowledge.
Below, this is given in the category ``Included, unlimited''
and to some extent also in ``Included, to be defined''.

\item
It serves as a set of guidelines that help decide whether a task is 
suitable for the International Olympiad in Informatics (IOI).
Based on this document, the International Scientific Committee (ISC)
evaluates the task proposals when selecting the competition tasks.

\item
As a consequence of the previous item, another purpose of the Syllabus 
is to help the organizers of national olympiads prepare
their students for the IOI.

\end{itemize}

The Syllabus aims to achieve these goals by providing a classification of topics and 
concepts from mathematics and computer science. 
More precisely, this Syllabus classifies each topic into one of six categories.
Ordered by topic suitability, these are:

\begin{myitemize}
\Iincluded Included, unlimited
\Idefine   Included, to be defined
\Icodeonly Included, not for task description
\Inofocus  Outside of focus
\Iexmaybe  Excluded, but open to discussion
\Iexcluded Explicitly excluded
\end{myitemize}

\noindent
In the next section we explain the scope of each category.

%}}}

\section{Categories}\label{sec:categories}%{{{

This Syllabus classifies a selection of topics into six different ca\-tegories. 
Obviously, such a set of topics can never be exhaustive.
Instead, the list given in this Syllabus should serve as examples that map out the boundary.
Topics not explicitly mentioned in the Syllabus should be classified as follows:
\begin{itemize}
\itemsep -3pt
\item Anything that is a prerequisite of an Included topic is also Included.
\item Anything that is an extension of an Excluded topic or similar to an Excluded topic is also Excluded.
\item Anything else that is not mentioned in the Syllabus is considered Outside of focus.
\end{itemize}

Note that issues related to the usage of suitable terminology and notations in competition tasks
are beyond the scope of this document.\footnote{See
T. Verhoeff: \emph{Concepts, Terminology, and Notations for IOI Competition Tasks},
\url{http://scienceolympiads.org/ioi/sc/documents/terminology.pdf}}

If there is a particular topic for which you are not sure how it should
be classified, we invite you to submit a clarification request to the 
current Syllabus maintainer.

\bigskip

\noindent
Here are the definitions of the six possible classifications:

\begin{description}
\item[\cincl\ Included, unlimited]~\\
    Topics in this category are considered to be prerequisite knowledge.
    Contestants are expected to know them. These topics can appear
    in task descriptions without further clarification.

    Example: \emph{Integer} in~\S\ref{subsubsec:NG}

\item[\cdefi\ Included, to be defined]~\\
    Contestants should know this topic,
    but when it appears in a task description,
    the statement should contain a sufficient definition.
    This category is usually applied in situations where a general
    concept that would be \cincl\ has many different ``flavors'' and a formal definition
    is needed to distinguish among those.

    Example: \emph{Directed graph} in~\S\ref{subsubsec:DS}~DS2

\item[\ccode\ Included, not for task description]~\\
    Topics that belong to this category should not appear in tasks 
    descriptions. However, developing solutions and understanding 
    model solutions may require the knowledge of these topics.

    Example: \emph{Asymptotic analysis of upper complexity bounds\/}
    in \S\ref{subsubsec:AL}~AL1

    Note: This is the main category that should be of interest when 
    preparing contestants for the IOI. 
    It should be noted that this set of topics
    contains a wide range of difficulties, starting from simple concepts and ending 
    with topics that can appear in problems that aim to distinguish among 
    the gold medallists. It is \textbf{not} expected that all contestants
    should know everything listed in this category.

\item[\cnfoc\ Outside of focus]~\\
    Any topic that is not explicitly addressed by the Syllabus
    should be considered to belong to this category.

    Contestants are not expected to have knowledge of these topics.
    Most competition tasks will not be related to any topics 
    from this category.

    However, this does not prevent the inclusion of 
    a competition task that is related to a particular topic
    from this category. The ISC may wish to include such a competition 
    task in order to broaden the scope of the IOI.
    
    If such a task is considered for the IOI,
    the ISC will make sure that the task can reasonably be solved
    without prior knowledge of the particular topic, and that 
    the task can be stated
    in terms of \cincl\ and \cdefi\ concepts in a precise, concise, 
    and clear way.

    Examples of such tasks being used at recent IOIs include:
    \begin{itemize}
    \itemsep -3pt
    \item Languages (a.k.a. Wikipedia) from IOI 2010 in Canada
    \item Odometer (a.k.a. robot with pebbles) from IOI 2012 in Italy
    \item Art class from IOI 2013 in Australia.
    \end{itemize}


\item[\cexcl\ Explicitly excluded]~\\
    Some of the harder algorithmic topics are explicitly marked as excluded.
    It is guaranteed that there will not be a competition
    task that \emph{requires} the contestants to know these areas.

    Furthermore, the tasks will be set with the goal that knowledge of 
    Excluded topics should not help in obtaining simpler solutions / solutions 
    worth more points.

    This category contains topics whose inclusion will result in
    problems that are unaccessible to a significant portion of IOI participants.
    This includes but is not limited to hard textbook algorithms and advanced
    areas in mathematics.

    Still, note that the Syllabus must not be interpreted to restrict in 
    any way the techniques that contestants are allowed to apply in solving 
    the competition tasks.
    
    Examples: \emph{Calculus\/} in~\S\ref{subsubsec:other-mathematics}

\item[\cemay\ Excluded, but open to discussion]~\\
    As the Syllabus is a living document, there can be cases when we consider
    bringing in some of the Excluded topics. Usually, the topics in question
    are natural extensions of Included topics, or ones where drawing an exact
    boundary is difficult. Should such topics appear, they will be temporarily
    classified as ``Excluded, but open to discussion'', and by doing so we encourage 
    all members of the IOI community to give us feedback on these topics.
\end{description}

\bigskip

\noindent
The rest of this document contains the classification of topics.
%}}}


\section {Számítástechnika, számítástudomány} % {Computing Science}
\label{subsec:computing-science}

\subsection {Progamozási Alapok (PA)} %{Programming Fundamentals (PF)}
\label{subsubsec:PF}

\subsubsection*{PA1. Alapvető programozási eszközök} % {PF1. Fundamental programming constructs}

\ctable {
    \cincl & \cincl & \cincl & Alapvető szintaxisa és szemantikája egy versenyen megengedett programozási nyelvnek
    \\ \hline % Basic syntax and semantics of a higher-level language
    \cincl & \cincl & \cincl & Változók, típusok, műveletek, kifejezések és értékadás
    \\ \hline % Variables, types, expressions, and assignment
    \cincl & \cincl & \cincl & Elágazások és ciklusok
    \\ \hline % Conditional and iterative control structures
    \cincl & \cincl & \cincl & Függvények és paraméterátadás
    \\ \hline % Functions and parameter passing
    \cincl & \cincl & \cincl & Egyszerű beolvasás és kiírás (A standard input/output ismerete kell.)
    \\ \hline % Simple I/O
    \cemay & \cemay & \cemay & Fájlba írás és olvasás
    \\ \hline
}


\subsubsection*{PA2. Alapvető adatszerkezetek} % {PF3. Fundamental data structures}

\ctable {
    \cincl & \cincl & \cincl & Elemi adattípusok (logikai, egész, karakter)
    \\ \hline % Primitive types (boolean, signed/unsigned integer, character)
    \cincl & \cincl & \cincl & Tömbök
    \\ \hline % Arrays (incl. multicolumn dimensional arrays)
    \cincl & \cincl & \cincl & Két vagy több dimenziós tömbök
    \\ \hline
    \cincl & \cincl & \cincl & Karekterláncok (string) és feldolgozásuk
    \\ \hline % Strings and string processing
    \ccode & \ccode & \ccode & Valós számok használata egyszerű és numerikusan stabil számításokra
    \\ \hline % Elementary use of real numbers in numerically stable tasks
    \ccode & \ccode & \ccode & A valós számok lebegőpontos reprezentációja, pontossági hibák léte.
        \footnote{Amikor csak lehetséges, a lebegőpontos számítások teljes elkerülése az előnyösebb megoldás.} % TODO: footnote nincs sehol.
    \\ \hline % \Icodeonly The floating-point representation of real numbers, the existence of precision issues.
        % \footnote{Whenever possible, avoiding floating point computations completely is the preferred solution.}
    \cemay & \ccode & \cincl & Mutatók (pointer) és referenciák
    \\ \hline % Pointers and references
    \cexcl & \cemay & \cincl & Láncolt adatszerkezetek
    \\ \hline % Linked structures
    \cemay & \cemay & \ccode & Törtszámok használata pontos számításokra
    \\ \hline % \Inofocus Using fractions to perform exact calculations.
    \cnfoc & \cnfoc & \cnfoc & Adatok memóriaképe
    \\ \hline % Data representation in memory
    \cnfoc & \cnfoc & \cnfoc & Dinamikus memóriafoglalás
    \\ \hline % {Heap allocation}
    \cexcl & \cexcl & \cexcl & Nemtriviális számítások lebegőpontos számokon, pontossági hibák kiküszöbölése
    \\ \hline % Non-trivial calculations on floating point numbers, manipulating precision errors
}

\subsubsection*{PA3. Rekurzió} %{PF4. Recursion}

\ctable {
    \cincl & \cincl & \cincl & A rekurzió fogalma
    \\ \hline % The concept of recursion
    \ccode & \cincl & \cincl & Rekurzív matematikai függvények
    \\ \hline % Recursive mathematical functions
    \cincl & \cincl & \cincl & Egyszerű rekurzív függvények (több függvény kölcsönös rekurziója is)
    \\ \hline % Simple recursive procedures (incl. mutual recursion)
    \cemay & \ccode & \ccode & A visszalépéses keresés rekurzív változata
    \\ \hline % Recursive backtracking
    \cexcl & \cemay & \cincl & Oszd meg és uralkodj stratégia
    \\ \hline % Divide-and-conquer strategies
}


\subsection {Algoritmusok és komplexitásuk (AL)} % {Algorithms and Complexity (AL)}
\label{subsubsec:AL}

\subsubsection*{AL1. Algoritmusok elemzése alapszinten} % Basic algorithmic analysis

\ctable {
    \cdefi & \cdefi & \cdefi & Algoritmus specifikáció, előfeltétel, utófeltétel, helyesség, invariánsok
    \\ \hline % Algorithm specification, precondition, postcondition, correctness, invariants
    \ccode & \ccode & \ccode & Aszimptotikus felső becslés a komplexitásra (lehetőleg nem formálisan)
    \\ \hline % Asymptotic analysis of upper complexity bounds (informally if possible)
    \cexcl & \ccode & \cdefi & $\mathcal{O}$ (ordó) jelölés a komplexitásra
    \\ \hline % Big O notation
    \ccode & \ccode & \ccode & Szokásos nagyságrendi osztályok: konstans, logaritmikus, lineáris, $\mathcal{O}(n \log n)$, négyzetes, köbös, exponenciális stb.
    \\ \hline % Standard complexity classes: constant, logarithmic, linear, $\mathcal{O}(n \log n)$, quadratic, cubic, exponential, etc.
    \ccode & \ccode & \ccode & Algoritmusok idő- és tárigényének optimalizálása
    \\ \hline % Time and space tradeoffs in algorithms
    \ccode & \ccode & \ccode & Hatékonyság mérése empirikusan
    \\ \hline % Empirical performance measurements.
    \cnfoc & \cnfoc & \cnfoc & kis ordó, omega és theta jelölések
    \\ \hline % Little o, Omega, and Theta notation
    \cnfoc & \cnfoc & \cnfoc & Paraméterek hangolása a futási idő vagy memória csökkentése érdekében.
    \\ \hline % Tuning parameters to reduce running time, memory consumption or other measures of performance
    \cexcl & \cexcl & \cexcl & Átlagos komplexitásra aszimptotikus becslések
    \\ \hline % Asymptotic analysis of average complexity bounds
    \cexcl & \cexcl & \cexcl & Rekurziós összefüggések használata komplexitás elemzéskor
    \\ \hline % Using recurrence relations to analyze recursive algorithms
}

\subsubsection*{AL2. Algoritmikus stratégiák} % AL2. Algorithmic strategies

\ctable {
    \ccode & \ccode & \ccode & Egyszerű ciklustervezéses stratégiák
    \\ \hline % Simple loop design strategies
    \ccode & \ccode & \ccode & Kimerítő keresés (brute force)
    \\ \hline % Brute-force algorithms (exhaustive search)
    \ccode & \ccode & \ccode & Mohó algoritmusok
    \\ \hline % Greedy algorithms
    \ccode & \ccode & \ccode & Dinamikus programozás
    \\ \hline % Dynamic programming
    \ccode & \ccode & \ccode & Rekurzív kiszámítás
    \\ \hline
    \ccode & \ccode & \ccode & Két mutató (2 pointers) technika
    \\ \hline
    \cemay & \ccode & \ccode & Bináris keresés
    \\ \hline
    \cemay & \ccode & \ccode & Visszalépéses keresés (backtrack, rekurzív és nem rekurzív is)
    \\ \hline % Backtracking (recursive and non-recursive),
    \cexcl & \ccode & \ccode & Elágazás és korlátozás
    \\ \hline % Branch-and-bound
    \cexcl & \ccode & \ccode & Oszd meg és uralkodj elv
    \\ \hline % Divide-and-conquer
    \cnfoc & \cnfoc & \cnfoc & Heurisztikák
    \\ \hline % Heuristics
    \cnfoc & \cnfoc & \cnfoc & Közelítő algoritmusok
    \\ \hline % Discrete approximation algorithms
    \cnfoc & \cnfoc & \cnfoc & Randomizált algoritmusok
    \\ \hline % Randomized algorithms.
    \cexcl & \cexcl & \cexcl & Klaszterező algoritmusok (pl. $k$-means)
    \\ \hline % Clustering algorithms (e.g. $k$-means, $k$-nearest neighbor)
    \cexcl & \cexcl & \cexcl & Többváltozós függvények szélsőérték keresése numerikus módszerekkel
    \\ \hline %Minimizing multi-variate functions using numerical approaches.
}

    \subsubsection*{AL3a. Algorithms}%{{{

    \begin{myitemize}
    \Icodeonly Simple algorithms involving integers: radix conversion, Euclid's algorithm, primality test by $\mathcal{O}(\sqrt{n})$ trial division, Sieve of Eratosthenes, factorization (by trial division or a sieve), efficient exponentiation
    \Icodeonly Simple operations on arbitrary precision integers (addition, subtraction, simple multiplication)\footnote{The necessity to implement these operations should be obvious from the problem statement.}
    \Icodeonly Simple array manipulation (filling, shifting, rotating, reversal, resizing, minimum/maximum, prefix sums, histogram, bucket sort)
    \Icodeonly Simple string algorithms (e.g., naive substring search)
    \Icodeonly\CC{sequential} processing/search \CC{and binary search}
    \Icodeonly \CC{Quicksort} and Quickselect to find the $k$-th smallest element.
    \Icodeonly\CC{$\mathcal{O}(n \log n)$} worst-case \CC{sorting algorithms (heap sort, merge sort)}
    \Icodeonly Traversals of ordered trees (pre-, in-, and post-order)
    \Icodeonly\CC{Depth- and breadth-first traversals}
    \Icodeonly Applications of the depth-first traversal tree, such as topological ordering and Euler paths/cycles
    \Icodeonly Finding connected components and transitive closures.
    \Icodeonly Shortest-path algorithms (Dijkstra, Bellman-Ford, Floyd-Warshall)
    \Icodeonly Minimum spanning tree (Jarn\'\i k-Prim and Kruskal algorithms)
    \Icodeonly $O(VE)$ time algorithm for computing maximum bipartite matching.
    \Icodeonly Biconnectivity in undirected graphs (bridges, articulation points).
    \Icodeonly Connectivity in directed graphs (strongly connected components).
    \Icodeonly Basics of combinatorial game theory, winning and losing positions, minimax algorithm for optimal game playing
    \end{myitemize}

    \begin{myitemize}
    \Iexmaybe Maximum flow. Flow/cut duality theorem.
    \end{myitemize}

    \begin{myitemize}
    \Iexcluded Optimization problems that are easiest to analyze using matroid theory. Problems based on matroid intersecions (except for bipartite matching).
    \Iexcluded Lexicographical BFS, maximum adjacency search and their properties
    \end{myitemize}
 
    %}}}
    \subsubsection*{AL3b. Data structures}%{{{
  
    \begin{myitemize}
    \Idefine Stacks and queues
    \Icodeonly \CC{Representations of graphs (adjacency lists, adjacency matrix)}
    \Idefine Binary heap data structures
    \Icodeonly Representation of disjoint sets: the Union-Find data structure.
    \Idefine Statically balanced binary search trees. Instances of this include binary index trees (also known as Fenwick trees)
    and segment trees (also known as interval trees and tournament trees).\footnote{Not to be confused with similarly-named data structures used in computational geometry.}
    \Icodeonly {Balanced binary search trees}\footnote{Problems will not be designed to distinguish between
    the implementation of BBSTs, such as treaps, splay trees, AVL trees, or scapegoat trees}
    \Idefine Augmented binary search trees
    \Icodeonly $O(\log n)$ time algorithms for answering lowest common ancestor queries in a static rooted tree.\footnote{Once again, different implementations meeting this requirement will not be distinguished.}
    \Icodeonly Creating persistent data structures by path copying.%\footnote{Gains in memory using methods such as fat-node will not be distinguished}}
    \Icodeonly Nesting of data structures, such as having a sequence of sets.
    \Idefine Tries
    \end{myitemize}

    \begin{myitemize}
    \Iexmaybe String algorithms and data structures: there is evidence that the inter-reducibility between KMP, Rabin-Karp hashing, suffix arrays/tree, suffix automaton, and Aho-Corasick makes them difficult to separate.
    \Iexmaybe Heavy-light decomposition and separator structures for static trees.
    \Iexmaybe Data structures for dynamically changing trees and their use in graph algorithms.
    \end{myitemize}

    \begin{myitemize}
    \Iexcluded Complex heap variants such as binomial and Fibonacci heaps,
    \Iexcluded Using and implementing \CC{hash tables} (incl. strategies to resolve collisions)
    \Iexcluded Two-dimensional tree-like data structures (such as a 2D statically balanced binary tree or a treap of treaps) used for 2D queries.
    \Iexcluded Fat nodes and other more complicated ways of implementing persistent data structures.
    \end{myitemize}

    %}}}
    \subsubsection*{AL4. Distributed algorithms}%{{{

    \begin{quote}
    This entire section is \cnfoc.
    \end{quote}

    %}}}
    \subsubsection*{AL5. Basic computability}%{{{

    \begin{quote}
    All topics related to computability are \cexcl.
    This includes the following:
    \CC{Tractable and intractable problems};
    \CC{Uncomputable functions};
    \CC{The halting problem};
    \CC{Implications of uncomputability}.
    
    However, see AL7 for basic computational models.
    \end{quote}

    %}}}
    \subsubsection*{AL6. The complexity classes P and NP}%{{{

    \begin{quote}
    Topics related to non-determinism, proofs of NP-hardness (reductions),
    and everything related is \cexcl.

    Note that this section only covers the results usually contained in
    undergraduate and graduate courses on formal languages and 
    computational complexity. The classification of these topics
    as \cexcl\ does not mean that an NP-hard 
    problem cannot appear at an IOI.
    \end{quote}

    %}}}
    \subsubsection*{AL7. Automata and grammars}%{{{

    \begin{myitemize}
    \Idefine Understanding a simple grammar in Backus-Naur form
    \end{myitemize}

    \begin{myitemize}
    \Inofocus Formal definition and properties of finite-state machines,
    \Inofocus Context-free grammars and related rewriting systems,
    \Inofocus Regular expressions
    \end{myitemize}
  
    \begin{myitemize}
    \Iexcluded Properties other than the fact that automata are graphs and that grammars have parse trees.
    \end{myitemize}
  
    %}}}
    \subsubsection*{AL8. Advanced algorithmic analysis}%{{{

    \begin{myitemize}
    \Icodeonly Amortized analysis.
    \end{myitemize}

    \begin{myitemize}
    \Inofocus Online algorithms
    \Inofocus Randomized algorithms
    \end{myitemize}

    \begin{myitemize}
    \Iexcluded Alpha-beta pruning
    \end{myitemize}
    
    %}}}
    \subsubsection*{AL9. Cryptographic algorithms}%{{{

    \begin{quote}
    This entire section is \cnfoc.
    \end{quote}

    %}}}
    \subsubsection*{AL10. Geometric algorithms}%{{{

    \begin{quote}
    In general, the ISC has a strong preference towards problems that can be solved using integer
    arithmetics to avoid precision issues. This may include representing some computed values as 
    exact fractions, but extensive use of such fractions in calculations is discouraged.

    Additionally, if a problem uses two-dimensional objects, the ISC prefers problems in which such objects are rectilinear.
    \end{quote}

    \begin{myitemize}
    \Icodeonly Representing points, vectors, lines, line segments.
    \Icodeonly Checking for collinear points, parallel/orthogonal vectors and clockwise turns (for example, by using dot products and cross products).
    \Icodeonly Intersection of two lines.
    \Icodeonly Computing the area of a polygon from the coordinates of its vertices.\footnote{The recommended way of doing so is to use cross products or an equivalent formula. TODO url}
    \Icodeonly Checking whether a (general/convex) polygon contains a point.
    \Icodeonly Coordinate compression.
    \Icodeonly $\mathcal{O}(n\log{n})$ time algorithms for convex hull
    \Icodeonly Sweeping line method
    \end{myitemize}
  
    \begin{myitemize}
    \Iexcluded Point-line duality
    \Iexcluded Halfspace intersection, Voronoi diagrams, Delaunay triangulations.
    \Iexcluded Computing coordinates of circle intersections against lines and circles.
    \Iexcluded Linear programming in 3 or more dimensions and its geometric interpretations.
    \Iexcluded Center of mass of a 2D object.
    \Iexcluded Computing and representing the composition of geometric transformations if the knowledge of linear algebra gives an advantage.
    \end{myitemize}

    %}}}
    \subsubsection*{AL11. Parallel algorithms}%{{{

    \begin{quote}
    This entire section is \cnfoc.
    \end{quote}

    %}}}


\section {Mathematics}
\label{subsec:mathematics}

\subsection {Arithmetics and Geometry}%{{{
\label{subsubsec:NG}

    \begin{myitemize}
    \Iincluded Integers, operations (incl.\ exponentiation), comparison
    \Iincluded Basic properties of integers (sign, parity, divisibility)
    \Iincluded Basic modular arithmetic: addition, subtraction, \\ multiplication
    \Icodeonly Prime numbers
    \Iincluded Fractions, percentages
    \Iincluded Line, line segment, angle, triangle, rectangle, square, circle
    \Iincluded Point, vector, coordinates in the plane
    \Iincluded Polygon (vertex, side/edge, simple, convex, inside, area)
    \Idefine Euclidean distances
    \Icodeonly Pythagorean theorem
    \end{myitemize}

    %{Excluded, but open to discussion}
    \begin{myitemize}
    \Iexmaybe Additional topics from number theory.
    \end{myitemize}
    
    %{Explicitly excluded\/}:
    \begin{myitemize}
    \Iexcluded geometry in 3D or higher dimensional spaces
    \Iexcluded analyzing and increasing precision of floating-point \\ computations
    \Iexcluded modular division and inverse elements
    \Iexcluded complex numbers,
    \Iexcluded general conics (parabolas, hyperbolas, ellipses)
    \Iexcluded trigonometric functions
    \end{myitemize}

%}}}
\subsection {Discrete Structures (DS)}%{{{
\label{subsubsec:DS}

    \subsubsection*{DS1. Functions, relations, and sets}

        \begin{myitemize}
        \Idefine\CC{Functions (surjections, injections, inverses, composition)}
        \Idefine\CC{Relations (reflexivity, symmetry, transitivity, equivalence relations,
        total/linear order relations, lexicographic order)}
        \Idefine\CC{Sets (inclusion/exclusion, complements, Cartesian products, power sets)}
        \end{myitemize}

        %{Explicitly excluded\/}:
        \begin{myitemize}
        \Iexcluded{Cardinality and countability} (of infinite sets)
        \end{myitemize}
  
    \subsubsection*{DS2. Basic logic}

        \begin{myitemize}
        \Iincluded First-order logic
        \Iincluded\CC{Logical connectives} (incl.\ their basic properties)
        \Iincluded\CC{Truth tables}
        \Iincluded\CC{Universal and existential quantification} (Note: statements should avoid definitions with nested quantifiers whenever possible.)
        \Icodeonly\CC{Modus ponens and modus tollens}
        \end{myitemize}

        %{Out of focus\/}:
        \begin{myitemize}
        \Inofocus \CC{Normal forms}
        \end{myitemize}
        
        %{Explicitly excluded\/}:
        \begin{myitemize}
        \Iexcluded \CC{Validity}
        \Iexcluded \CC{Limitations of predicate logic}
        \end{myitemize}

    \subsubsection*{DS3. Proof techniques}

        \begin{myitemize}
        \Idefine\CC{Notions of implication, converse, inverse, contrapositive, negation, and contradiction}
        \Icodeonly\CC{Direct proofs, proofs by: counterexample, contraposition, contradiction}
        \Icodeonly\CC{Mathematical induction}
        \Icodeonly\CC{Strong induction} (also known as complete induction)
        \Iincluded\CC{Recursive mathematical definitions} (incl.\ mutually recursive definitions)
        \end{myitemize}

    \subsubsection*{DS4. Basics of counting}

        \begin{myitemize}
        \Iincluded\CC{Counting arguments (sum and product rule, arithmetic and geometric progressions, Fibonacci numbers)}
        \Idefine\CC{Permutations and combinations (basic definitions)}
        \Idefine Factorial function, binomial coefficients
        \Icodeonly\CC{Inclusion-exclusion principle}
        \Icodeonly\CC{Pigeonhole principle}
        \Icodeonly\CC{Pascal's identity}, \CC{Binomial theorem}
        \end{myitemize}

        %{Explicitly excluded\/}:
        \begin{myitemize}
        \Iexcluded Solving of recurrence relations
        \Iexcluded Burnside lemma
        \end{myitemize}
  
    \subsubsection*{DS5. Graphs and trees}

        \begin{myitemize}
        \Idefine\CC{Trees} and their basic properties, rooted trees
        \Idefine\CC{Undirected graphs} (degree, path, cycle, connectedness, Euler/Hamil\-ton path/cycle, handshaking lemma)
        \Idefine\CC{Directed graphs} (in-degree, out-degree, directed path/cycle, Euler/Hamilton path/cycle)
        \Idefine\CC{Spanning trees}
        \Idefine\CC{Traversal strategies}
        \Idefine `Decorated' graphs with edge/node labels, weights, colors
        \Idefine Multigraphs, graphs with self-loops
        \Idefine Bipartite graphs
        \Icodeonly Planar graphs
        \end{myitemize}

        %{Explicitly Excluded}
        \begin{myitemize}
        \Iexcluded Hypergraphs
        \Iexcluded Specific graph classes such as perfect graphs
        \Iexcluded Structural parameters such as treewidth and expansion
        \Iexcluded Planarity testing
        \Iexcluded Finding separators for planar graphs
        \end{myitemize}

    \subsubsection*{DS6. Discrete probability}
  
        \begin{quote}
        Applications where everything is finite (and thus arguments about probability can be easily
        turned into combinatorial arguments) are \cnfoc, everything more complicated
        is \cexcl.
        \end{quote}

%}}}
\subsection {Other Areas in Mathematics}%{{{
\label{subsubsec:other-mathematics}

    %{Explicitly excluded\/}:
    \begin{myitemize}
    \Iexcluded Geometry in three or more dimensions.
    \Iexcluded  Linear algebra, including (but not limited to):
        \begin{myitemize}
        \item Matrix multiplication, exponentiation, \\
              inversion, and Gaussian elimination
        \item Fast Fourier transform
        \end{myitemize}
    \Iexcluded Calculus
    \Iexcluded Theory of combinatorial games, e.g., NIM game, Sprague-Grundy theory
    \Iexcluded Statistics
    \end{myitemize}

%}}}


\end{document}
